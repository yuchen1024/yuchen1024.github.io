%!TEX program = xelatex
\newcommand{\path}{/Users/chenyu/Documents/Research/PaperEngine}
\documentclass[a4paper,12pt]{ctexart}
%\newcommand{\keywords}[1]{\par\addvspace\baselineskip\noindent\keywordname\enspace\ignorespaces#1}

\input{\path/mypapersetting}

\begin{document}

\title{高级密码组件及其应用\\教学大纲}

\author{
    陈宇\\ 
    山东大学\\
    \blue{\texttt{cycosmic@gmail.com}}
}

\date{}

\maketitle

\section*{课程信息}
\begin{trivlist}
\item 课时/学分: 32/2     
\item 课程属性: 专业选修课
\item 主讲教师:陈宇
\item 中文名称: 高级密码组件及其应用
\item 英文名称:Advanced Cryptographic Primitives and Their Applications

\item 教学目的、要求\\ 
    本课程重点介绍理论密码学中的高级密码组件, 
    旨在使学生精通若干高级密码组件(包括各类高级单向函数、程序混淆和受限伪随机函数、不可延展非交互式证明、哈希证明系统等)的概念和构造, 并掌握其在公钥密码学中的重要应用。本课程旨在帮助学生在《理论密码学》课程之上进一步拓宽加深密码理论基础, 追踪科研前沿进展. 

\item 预修课程:理论密码学 (强烈建议选课同学课前阅读相关参考文献)
\end{trivlist}

\section*{主要内容}
\begin{trivlist}
\item 第一讲: 公钥加密简介(学时:3+1) ~\cite{NS-CRYPTO-2009,DFMV-ASIACRYPT-2013,Wee-PKC-2016}
\begin{itemize}
    \item 公钥加密的KEM-DEM构造方法 
    \item 公钥加密的传统安全
    \item 超越传统语义安全: 抗泄漏安全、抗篡改安全、消息相关密钥安全
\end{itemize}


\item 第二讲: 高级单向函数I(学时:3+1) ~\cite{PW-STOC-2008,Chen-CT-RSA-2018}
\begin{enumerate}
    \item 有损陷门函数的概念、构造
    \item 有损陷门函数的应用
    \item 有损陷门函数的重要扩展
\end{enumerate}

\item 第三讲: 高级单向函数II(学时:3+1) ~\cite{RS-TCC-2009,KMO-EUROCRYPT-2010}
\begin{enumerate}
    \item 相关积陷门单向函数的概念与构造
    \item 自适应单向陷门函数的概念与构造
    \item 自适应单向陷门函数的应用
\end{enumerate}

\item 第四讲: 非交互式零知识证明及其应用(学时:3+1) ~\cite{NY-STOC-1990,DDN-STOC-1991,Sahai-FOCS-1999}
\begin{enumerate}
    \item Naor-Yung双重加密范式
    \item Dolev-Dwork-Naor构造
    \item 不可延展非交互式零知识证明及其应用
\end{enumerate}

\item 第五讲: 哈希证明系统及其应用(学时:3+1) ~\cite{CS-EUROCRYPT-2002,QL-ASIACRYPT-2013,Wee-PKC-2016}
\begin{enumerate}
    \item 哈希证明系统的定义及构造
    \item 哈希证明系统在CCA安全中的应用
    \item 哈希证明系统在KDM安全和抗泄漏安全中的应用
\end{enumerate}

\item 第六讲: 可提取哈希证明系统及其应用(学时:3+1) ~\cite{Wee-CRYPTO-2010}
\begin{enumerate}
    \item 可提取哈希证明系统的定义及构造
    \item 可提取哈希证明系统在CCA安全中的应用
    \item 自适应单向陷门关系
\end{enumerate}

\item 第七讲: 程序混淆与受限伪随机函数(学时:3+1) ~\cite{Barak-CRYPTO-2001,BW-ASIACRYPT-2013,SW-STOC-2014}
\begin{enumerate}
    \item 程序混淆的概念与构造
    \item 受限伪随机函数的概念
    \item 程序混淆与受限伪随机函数的应用
\end{enumerate}

\item 第八讲: 可公开求值伪随机函数(学时:3+1) ~\cite{Chen-SCN-2014,Chen-ASIACRYPT-2018}
\begin{enumerate}
    \item 可公开求值伪随机函数的概念与构造
    \item 可公开求值伪随机函数的应用
    \item 可穿孔可公开求值伪随机函数及其在抗泄漏密码学中的应用
\end{enumerate}
\end{trivlist}


% 4、  Levin的通用单向函数
% 5、  弱单向函数隐含强单向函数
% 第三周: 单向函数II (3学时)主讲: 薛锐
% 1、  基于各类具体假设的单向函数构造
% 2、  单向性 vs. 伪随机性范式—Goldreich-Levin定理
% 第四周: 不可区分性与困难假设 (3学时)主讲: 陈宇
% 1、  渐进复杂度
% 2、  (完美/统计/计算)不可区分性
% 3、  Hybrid Lemma和Composition Lemma
% 第五周: 伪随机数发生器 (3学时)主讲: 陈宇
% 1、  伪随机性与伪随机数发生器
% 2、  介伪随机数发生器的通用构造(expansion theorem)与具体构造
% 3、  伪随机数发生器的重要应用
% 第六周:伪随机函数(3学时)主讲: 陈宇
% 1、  伪随机函数的定义与安全性
% 2、  伪随机函数的构造: Goldreich-Goldwasser-Micali类和Naor-Reingold类构造
% 3、  伪随机函数各种变体
% 4、  伪随机函数的重要应用以及与学习理论之间的关系
% 第七周: 消息验证码 (3学时)主讲: 陈宇
% 1、  消息验证密码的定义及典型应用
% 2、  信息论和计算意义下的构造
% 第八周: 私钥加密(3学时)主讲: 陈宇
% 1、  私钥加密的定义以及选择明文安全和选择密文安全
% 2、  私钥加密的通用构造
% 3、  认证加密的定义、构造及应用
% 第九周: 哈希函数 (3学时)主讲: 陈宇
% 1、  哈希函数及各类安全性质
% 2、  哈希函数的通用构造以及具体构造
% 3、  哈希函数的典型应用,包括Proof of Work in Bitcoin和Verifiable Storage via Merkle Hash Tree
% 第十周: 数字签名(3学时)主讲: 陈宇
% 1、  数字签名的定义及安全性
% 2、  基于单向函数的通用构造
% 第十一周: 随机预言机模型(3学时)主讲: 陈宇
% 1、  随机预言机模型的方法及最新研究进展
% 2、  随机预言机模型中的归约和证明技巧
% 3、  随机预言机的实例化方法
% 第十二周: 公钥加密 (3学时)主讲: 陈宇
% 1、  公钥加密的定义及选择明文安全
% 2、  选择明文安全的一般构造和通用构造
% 3、  选择密文安全及通用构造, 包括Naor-Yung构造和Dolev-Dwork-Naor构造
% 第十三周: 复杂性理论 (3学时)主讲: 陈宇
% 1、  图灵机计算模型
% 2、  平均、最差复杂性
% 3、  常见复杂性类
% 第十四周:  零知识证明系统 (3学时)主讲: 陈宇
% 1、  交互式证明系统
% 2、  (非交互式)零知识证明系统
% 3、  零知识证明系统的应用

\bibliographystyle{alpha}
\bibliography{\path/bib/mycrypto}
\end{document}
